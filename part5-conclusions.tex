
\part{Conclusions}
\label{partV}

\chapter{Outcome}
\glsresetall
\label{chapter:summary}

This dissertation deals with fundamental topics of Data Mining under the constraints of high-dimensional data streams. We addressed them from the most specific to the most general: Estimating Dependency (\hyperlink{Q1}{\textbf{Q1}}), Monitoring (\hyperlink{Q2}{\textbf{Q2}}) and Knowledge Discovery (\hyperlink{Q3}{\textbf{Q3}}). 

Those topics are intertwined: virtually any Knowledge Discovery task benefits from estimating dependency (c.f. Section \ref{sec:centralrole}). However, in the high-dimensional streaming setting, one also needs solutions for the efficient monitoring of such estimates. Data Mining in such context is difficult because one must simultaneously address two orthogonal challenges: high-dimensionality and data streams (c.f. Sections \ref{challenges:highdimensionality} and \ref{challenges:datastreams}). 

In Part \ref{partII}, we addressed the challenges of estimating dependency (\hyperlink{Q1}{\textbf{Q1}}) in high-dimensional data streams. After identifying the constraints and requirements, we introduced \gls{MCDE}, a framework to estimate multivariate dependency in heterogeneous data streams. In a nutshell, \gls{MCDE} quantifies dependency as the statistical discrepancy between marginal and conditional distributions over multiple Monte Carlo simulations. Based on different statistical test, we derived three new estimators: \gls{KSP}, \gls{MWP} and \gls{CSP}, and showed that they fulfil all the requirements described earlier. Compared to other approaches, \gls{MCDE} provides high statistical power on a large panel of dependencies, while being very efficient. Furthermore, we introduced index operations for the streaming setting and illustrated the benefits of our framework against a real-world use case: the \acrshort{Bioliq} power plant. Due to its anytime nature and efficient index structure, \gls{MCDE} gives way to efficient monitoring of dependency in data streams. 

\glsunset{Bioliq}
Then, we generalised the task of monitoring (\hyperlink{Q2}{\textbf{Q2}}) statistics in Part \ref{partIII}. We proposed a novel bandit model, named the \gls{S-MAB}, which captures the efficiency trade-off that is central to many real-world applications. We presented a new algorithm, \gls{S-TS}, which combines \gls{MP-TS} with a new procedure to decide on the number of arms played per round, a so-called `scaling policy'. Our analysis and experiments showed that it enjoys strong theoretical guarantees and excellent empirical behaviour. We also proposed an extension of our algorithm for the non-static setting, by combining it with \gls{ADWIN}, a state-of-the-art change detector. We illustrated with the example of \gls{Bioliq} that one can use our algorithms to monitor multiple statistics online.   

In Part \ref{partIV}, we addressed Knowledge Discovery (\hyperlink{Q3}{\textbf{Q3}}) in high-dimensional data streams. While the knowledge of dependencies in streams already is valuable as such, we illustrated the impact of efficient dependency monitoring systems on downstream Data Mining tasks. First, we achieved Subspace Search in Data Streams (Chapter \ref{chapter:subspacesearch}) by proposing a new algorithm, \gls{SGMRD}, that leverages \gls{MCDE} and \gls{S-MAB} algorithms. We  showed that  \gls{SGMRD} leads to state-of-the-art performance against a typical Data Mining task: the detection of outliers. 
Then, we looked at a specific application: detecting text outlier from large corpora (Chapter \ref{chapter:textoutlier}). This task is particularly challenging because text outliers are manifold, domain-specific, and the task is unsupervised in nature. We are the first to make the distinctions between two types of outliers: out-of-distribution (Type O) and misclassified (Type M) documents. We proposed an approach which simultaneously detects both types of outliers. Our algorithm, \gls{kj-NN}, leverages self-supervision signals from an initial but imperfect classification. Our experiments show that our approach outperforms each competitor and baseline w.r.t.\ both outlier types while delivering interpretable results. Using our results may increase the quality of massive text archives by much and assist the annotators.  

All in all, this dissertation presents fundamental contributions to the field of Data Mining, focusing on the particularly challenging setting of high-dimensional data streams. Our contributions have been well-received in peer-review conferences: \gls{MCDE} \cite{DBLP:conf/ssdbm/FoucheB19} was presented at the 31st International Conference on Scientific and Statistical Database Management (SSDBM'19) and received the `Best Paper Award'. We published an extended study of the \gls{MCDE} framework \cite{DBLP:conf/ssdbm/FoucheBMKB20} in the `Distributed and Parallel Databases' (DAPD) journal (Springer). We presented the \gls{S-MAB} \cite{DBLP:conf/kdd/FoucheKB19} algorithms at the 25th ACM SIGKDD Conference on Knowledge Discovery and Data Mining (KDD'19), and \gls{kj-NN} \cite{DBLP:conf/ssdbm/FoucheMGZBH20} at the 20th IEEE International Conference on Data Mining (ICDM'20). 

Our contributions raised several interesting questions, in particular about their impact on downstream analysis tasks. While we did make a few strides towards answering those questions with \gls{SGMRD} \cite{DBLP:conf/review/FoucheKB20} and \gls{kj-NN} \cite{DBLP:conf/ssdbm/FoucheMGZBH20}, we elaborate on the open problems and future challenges of this dissertation hereafter. 

\chapter{Future Work}
\glsresetall
\label{chapter:futurework}

Our contributions raise several interesting questions. While we partially addressed some of them, the required research efforts extend beyond the scope of this dissertation. We identify the following open research directions: 

\textbf{Multi-Scale Dependency Monitoring:} While we showed that \gls{MCDE} (Chapter \ref{chapter:MCDE}) leads to efficient monitoring of dependency in streams, e.g., via exponential weighting, \gls{MCDE} could benefit from a more flexible update mechanism, using for instance a sliding window of adaptive size \cite{DBLP:conf/sdm/BifetG07}. Breaking free from the fixed-size sliding window model may help to generalise dependency estimation to time contexts of various scales. While doing so, one may need to investigate the integration of newer statistical test into the \gls{MCDE} framework, such as \cite{fligner1981robust} or \cite{brunner2000nonparametric}.  

\textbf{Regret Analysis of the Non-static Scaling Multi-Armed Bandit:} In Part \ref{partIII}, we solved the \gls{S-MAB} problem in the static setting with \gls{S-TS}. For the non-static environment, we proposed an improvement based on \gls{ADWIN}. Our algorithm performs better than the existing approaches empirically. However, the static regret analysis already is quite involved, and extending it to non-static environments is a challenge. We hypothesise that this success is due to a class of non-stationarity that our solution exploits -- but which we have not formalised yet. A first step is to leverage the theoretical guarantees of \gls{ADWIN} to derive general guarantees for Scaling Bandits with \gls{ADWIN}. Still, the technical difficulty arises in guaranteeing that change points are accurately detected. For the analysis, non-trivial modifications of the \gls{ADWIN} algorithm might be necessary. 

\textbf{Anytime Trade-off Strategies:} We proved in Chapter \ref{chapter:MCDE} that the estimates from \gls{MCDE} converge as one increases the number of simulations. This is a crucial characteristic of \gls{MCDE} that we call \textit{anytime flexibility}. Based on a computational budget, users can bound the estimate quality. Conversely, one can interrupt the computation whenever the estimate reaches a desired quality level. The same year, \cite{DBLP:conf/edbt/VollmerB19} published an estimator for \gls{MI} with the same characteristic. With the popularisation of edge computing, reducing the cost of basic tasks -- e.g., dependency estimation -- is very appealing, because computation typically is distributed among systems with limited resources. However, in Knowledge Discovery, one is often interested in finding sets of attributes with high dependency. In other words, the value of each estimate, which is unknown a priori, may not all be equally interesting to the end-user. The question is then how to distribute in an anytime fashion a global computational budget among concurrent estimates? While our contribution \cite{DBLP:conf/ssdbm/FoucheB19} and  \cite{DBLP:conf/edbt/VollmerB19} addressed this question for individual dependency estimates, generalising the optimisation of anytime algorithms w.r.t. multiple problems remains open. Now, it would be interesting to refine bounds, like the one presented in Theorem \ref{"th:hoeffding-chernoff-contrast"}, via further assumptions, and to leverage user-specific criterion to guide the allocation of resources between concurrent problems.

\textbf{High-Dimensional Stream Mining with User Constraints:} In the literature, integrating user constraints into Data Mining algorithms often is seen as a complication \cite{DBLP:journals/computer/HanLN99}. With the \gls{S-MAB} model (Chapter \ref{chapter:S-MAB}), we have seen that user information (such as a threshold on the interestingness of estimates) may also help to reduce computational costs drastically. Integrating such information into algorithms might be an essential aspect to tackle Data Mining in high-dimensional data streams. This idea is in line with our future work on Anytime Trade-off Strategies (see above). The question is how to formally integrate user constraints into mining algorithms, in particular, considering the type of information one may receive from users with different levels of expertise?

\textbf{Mining Causality:} With \gls{MCDE} (Chapter \ref{chapter:MCDE}) and \gls{S-MAB} (Chapter \ref{chapter:S-MAB}), one can estimate the dependence between attributes in high-dimensional data sets and discover interesting, previously unknown associations. The next step is to ask, for each of those associations, whether they emerge from a causal relationship, or turn out to be spurious. Recent advances in a related field, Causal Inference \cite{pearl2009, nandy2016causal, lopezpaz2016dependence, DBLP:journals/cacm/Pearl19}, provide elements to answer this question solely based on observational data. While passive causal inference methods are somewhat limited, because they require relatively strong assumptions, it would be interesting to extend the \gls{MCDE} framework in this direction. Our intuition is that slight modifications of our subspace slicing scheme may allow testing the existence of confounding variables for a given association. One could then build a so-called `dependency network' \cite{DBLP:journals/jmlr/HeckermanCMRK00, DBLP:journals/ml/SchulteQKYS16, doi:10.1146/annurev-statistics-060116-053803}, i.e., a graphical representation of dependency in the stream, providing new insights to end-users as well. By extending our methods, we could mine such networks in real-time in high-dimensional data streams. 

\textbf{Analysis of Multivariate Times Series Embeddings:} In Chapter \ref{chapter:textoutlier}, we showed that exploiting local neighbourhoods in word/document embeddings helps to detect outlying text. Integrating the methods from Chapter \ref{chapter:subspacesearch} may bring an incremental improvement in the static setting already, and solve similar tasks in massive streams of text. Perhaps even more interesting is to transfer our approach to other domains, e.g., multivariate times series. For example, we could project sequences of measurements from the \acrshort{Bioliq} power plant into an embedding space. Then, using \gls{kj-NN}, we may automatically detect normal and abnormal states in the plant, and, with \gls{SGMRD}, adapt to changes in the streaming setting. In predictive maintenance, there typically are only very few labels, such setting is known as `weakly-supervised' \cite{10.1093/nsr/nwx106, DBLP:conf/cikm/MengSZ018}.
It would be interesting to investigate extensions of our approach to handle weak supervision or further complications. %, multi-label or hierarchical classification.

To conclude, this dissertation advances the state of the art of the Data Mining field. We provided fundamental contributions with significant impacts on the general task of Knowledge Discovery in high-dimensional data streams. Last but not least, our work paves the way for multiple exciting future research topics.   